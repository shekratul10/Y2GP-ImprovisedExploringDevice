\section{Evaluation}
A rover was designed so as it is able to autonomously navigate through a maze using an outer wall following approach, wall following was done based on light intensity coming from the walls of the maze, this was then used as a basis to perform the mapping of the maze where the light intensity would also tell us about any features of the maze present such as junctions or corners. Using this information it was possible to represent the mapping using a graph data structure which made the process of performing any maze solving algorithm easier to implement, the algorithm of choice was dijkstra’s with heaps due to its simplicity and relative low computing cost. This implementation covers two out of the three functional requirements. The balancing was done on a separate rover where an accelerometer and gyroscope module were used in conjunction with a pid control system to be able to balance the robot on two wheels, covering the final requirement of the project.

Many challenges were encountered during the completion of the project. Multiple approaches were tested, such fulfilling all three functional requirements on one rover, increasing the role of computer vision in mapping by implementing edge, contour and corner detection; streaming the video feedback onto the server so that opencv could be used for computer vision. However the group experienced many roadblocks, such as time constraints, complexity of some of the problems and poor computational performance, the amalgamation of these constraints is what led to the final design and implementation of the rover.